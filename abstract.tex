\renewcommand{\abstractnamefont}{\normalfont\Large\bfseries}
%\renewcommand{\abstracttextfont}{\normalfont\Huge}

\begin{abstract}
\hskip7mm

\begin{spacing}{1.3}
 
 
``Dans le cadre de collaborations avec le laboratoire CLLE-ERRSàB de l'université Bordeaux Montaigne et l'université de Waseda à Tokyo (Japon), nous (le Labri ndlr) étudions les situations de communication intra et inter culturelles, en considérant les informations sonores et visuelles. Nous avons d'ores et déjà constitué un corpus audio-vidéo multilingue (français, japonais, anglais américain et portugais du brésil) pour un ensemble de situtations prédéfinies (admiration, séduction, arrogance, doute, irritation, évidence, politesse, surprise, ...).'' --- Jean-Luc Rouas

Ce projet vise à étudier comment un utilisateur peut construire à partir d'un son (exprimant par exemple l'admiration) et d'une vidéo (exprimant par exemple le mépris) un document audio-vidéo exprimant par exemple l'ironie. L'utilisateur peut être amené à utiliser les données enregistrées dans sa langue natale ou non, dans le but d'étudier les correspondances entre les langues.

Notre but est donc de réaliser une interface graphique permettant de réaliser des tests prosodiques\footnote{Prosodie : ``La prosodie (ou la prosodologie) est une branche de la linguistique consacrée à la description (aspect phonétique) et à la représentation formelle (aspect phonologique) des éléments de l’expression orale tels que les accents, les tons, l’intonation et la quantité, dont la manifestation concrète dans la production de la parole, est associée aux variations de la fréquence fondamentale (F0), de la durée et de l’intensité (paramètres prosodiques physiques). Ces variations étant perçues par l’auditeur comme des changements de hauteur (ou de mélodie), de longueur et de sonie (paramètres prosodiques subjectifs)''.\cite{di2000} } sur des cobayes.
Ces sujets auront à faire un choix, d’une vidéo parmi plusieurs, à fusionner avec une bande son parmi une autre liste.
Ce mixe de vidéo/son devra donner une réponse à une question du genre : “Réaliser une vidéo qui exprime l’ironie.”

\end{spacing}
\end{abstract}
