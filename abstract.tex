\renewcommand{\abstractnamefont}{\normalfont\Large\bfseries}
%\renewcommand{\abstracttextfont}{\normalfont\Huge}

\begin{abstract}
\hskip7mm

\begin{spacing}{1.3}
 
 
Ce projet nous permet de comprendre la prosodie selon le pays d'origine puisque nous avons eu la chance d'être en contact avec des personnes venant  de l’université Bordeaux Montaigne ainsi que de l’université de Waseda à Tokyo. La langue de chaque pays est déjà définie dans un corpus créé par des chercheurs français, japonais, anglais et portugais du Brésil pour des situations prédéfinies. 

Ce projet vise à étudier la façon dont un utilisateur peut construire un document audio-vidéo à partir d’un son  et d’une vidéo. Par exemple, une vidéo exprimant l’admiration et un fichier audio exprimant le mépris servent à exprimer l’ironie. L’utilisateur peut être amené à manipuler les données enregistrées dans sa langue natale ou dans une autre, dans le but d’étudier les correspondances entre les langues. 

Notre but est donc de réaliser une interface graphique permettant d'effectuer des tests prosodiques\footnote{Prosodie : ``La prosodie (ou la prosodologie) est une branche de la linguistique consacrée à la description (aspect phonétique) et à la représentation formelle (aspect phonologique) des éléments de l’expression orale tels que les accents, les tons, l’intonation et la quantité, dont la manifestation concrète dans la production de la parole, est associée aux variations de la fréquence fondamentale (F0), de la durée et de l’intensité (paramètres prosodiques physiques). Ces variations étant perçues par l’auditeur comme des changements de hauteur (ou de mélodie), de longueur et de sonie (paramètres prosodiques subjectifs)''.\cite{di2000} } sur des cobayes.
L’administrateur peut lister, ajouter ou supprimer du contenu dans la base de données du test pour pouvoir, s’il le souhaite, mettre à jour celle-ci. Les utilisateurs auront, quant à eux, à inscrire des informations personnelles pour participer aux tests et auront par la suite le choix entre deux modes. Ils pourront choisir entre un mode entrainement pour se familiariser avec le logiciel et entre un mode test. Dans ces deux modes, les utilisateurs auront à faire le choix d’une vidéo parmi plusieurs à fusionner avec une bande son qui se trouve dans une autre liste.
Ce mix de vidéo/son devra donner une réponse à une question du type : “Réaliser une vidéo qui exprime l’ironie.”. A la fin de chaque test, les données utilisateurs et les réponses aux questions seront extraites pour pouvoir réaliser des statistiques.

\end{spacing}
\end{abstract}
