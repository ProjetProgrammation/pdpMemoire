\chapter{Bilan}


L'UE de projet de programmation a été important lors de cette première année de master. En effet, elle nous a donné la possibilité de mettre en application les connaissances accumulées durant nos années d'études précédente dans l'élaboration d'un projet. cependant l'étude de l'existant ainsi que la documentation ont été des parties nouvelles pour nous tous.\\
Cette projet nous permet d'avoir une ligne supplémentaire dans nos CV respectifs. Plus qu'une mise en situation, c'est, en soit, une expérience professionnelle, puisque nous avions un client à satisfaire.\\
On peut aussi souligner l'expérience de travail en équipe, qui induit une bonne répartition des tâches, ainsi que l'utilisation d'une gestionnaire de version (dans notre cas, \textit{Github}).\\

Cependant, notre application n'est pas optimal et plusieurs poursuites de projet peuvent être envisagées.\\
La sauvegarde et la reprise des test en cas de crash de l'application serait une poursuite à mettre en place dans l'avenir. En effet, notre application extrait les données utilisateur ainsi que ses réponses en fin de test. Malheureusement à cause de ce système, un crash ne permet pas d'avoir une sauvegarde préalable. La solution serait d'écrire dans le fichier texte à chaque validation de réponses. Ensuite en cas d'arrêt non voulu de l'application, un algorithme serait mis en place à partir du fichier créer qui permettrait de retrouver le nombre de question restante.\\
La seconde amélioration possible serait l'ajout de la synchronisation audio-vidéo. Nos recherches nous ont mené sur deux pistes différentes. La première était l'utilisation de la bibliothèque xuggle. Elle utilise les bibliothèques FFmpeg permettant d'encoder, de décoder des fichiers audio ou vidéo. Son utilisation est cependant impossible du fait que le projet ait été abandonné et rendant par la même occasion la synchronisation mal venu. En effet aucun support n'est disponible via cette bibliothèque si besoin.\\
Par la suite nos recherches nous ont amené à étudier la bibliothèque Opencv. Elle est la référence dans le domaine du traitement d'images. Comprenant de nombreuses fonctionnalités elle possède celles que nous recherchions, c'est à dire la lecture, l'écriture et l'affichage d’une vidéo. Elle nous permettrait notamment la modification du nombre d'image par seconde de notre fichier vidéo pour pouvoir avoir une bonne synchronisation en prenant pour référence notre fichier audio.\\

Nous avons pensé à un algorithme pour pouvoir mixer les deux fichiers. Il a été pensé sur papier mais n'a pas été mis en œuvre.

\begin{verbnobox}[\small]

Fonction Synchronisation( chemin video, chemin audio) :

Vérification lecture et ouverture fichier Video
Vérification lecture et ouverture  fichier Audio

Création du fichier de sortie mix

Récupération des Streams des fichiers audio et video

Récupération de la StreamAudio pour le fichier audio et récuperation de la Stream Video pour le fichier vidéo

Si temps_StreamAudio < temps_streamVideo alors on augmente frame_rate_StreamVideo jusqu'à temps_StreamAudio = temps_streamVideo.
Si temps_StreamAudio > temps_streamVideo alors on diminue le  frame_rate_StreamVideo jusqu'à temps_StreamAudio = temps_streamVideo.

Ajout des streams Selectionnées dans le fichier de sortie

Vérification des Streams Selectionnées

Si aucune erreur alors 

	Tant qu' il reste des datas à lire dans le Stream Video ou dans le streamAudio
		
		si les datas appartienne au fichier Video
		
			alors tant qu'une image n'est pas complète on additionne les packets
			si complète on l'ajoute à notre fichier de sortie


		si les datas appartienne au fichier Audio
		
			alors tant qu'un échantillon de la piste audio n'est pas complet on additionne les packets

			si complet on l'ajoute à notre fichier de sortie

\end{verbnobox}

Stream : Un ensembles de données accessibles dans le temps. Les éléments d'un stream sont appelé les Frames et chaque Frame est encodé par une sorte différentes de codec
Packets : Ce sont des morceaux de données qui contiennent des bits de données qui sont décodés dans des frames pour que finalement nous puissions les manipuler.