\chapter{Analyse de l'existant}

 Au lieu de "l'article XX nous donne la définition de truc" (aucune
information apportée au lecteur) dire "Le phénomène truc décrit
comment le cerveau humain extrait un sens à partir du flux visuel, et a
permis de comprendre comment nous arrivons à différencier la poule
d'un œuf (article XX)." Là le lecteur apprend quelque chose et /si/ il
  veut en savoir plus ou vérifier, il ira consulter l'article qui vous
sert de référence et de justification. Ça doit être comme ça /pour
tout/


\section{Compréhension du sujet}\label{comprehension_sujet}

La prosodie est un terme complexe mettant en exergue les liens entre les expressions (faciales et vocales) et les émotions qui en découlent néanmoins nous ne pouvons réduire cette définition à ce simple résumé, de ce fait l'article \cite{bachorowski1999 vocaux} nous permet d'obtenir une définition plus précise sur ce terme technique.

Ainsi, la voix est un élément important de la prosodie car elle est un acteur primordial de la transmission d'émotions mais en fonction de nos origines sociales culturelles cette perception peut différer \cite{auberge2002prosodie}.
De plus la voix se combine avec les expressions faciales qui sont le portrait de l'émotion que nous voulons transmettre. Les changements que nous effectuons sur notre visage(position des sourcils, lèvres...) permet l'expression d'émotions par le visage \cite{ekman2003unmasking} .

Enfin, le document \cite{fourer:hal-00992083} nous offre une approche sociétale du problème avec une étude sur la prosodie attitudinale pour la langue japonaise.


\section{Références sur la transformation de la voix}\label{ref_transfo_voix}

Pour les études prosodiques, des modulations, découpages et modifications en tous genres de la voix sont nécessaires mais la voix est un mécanisme complexe comme nous il nous l'est présenté dans le cours de Ricardo Gutierrez-Osuna \cite{Gutierrez-Osuna:ISP-PMS} et l'article de l'IEEE Signal Processing Letters \cite{haagen1994transformation}.

L'extrait de la IEEE International Conference on Acoustics de 1998 \cite{Acero:ICASSP98-II-881} décrit deux possibles modifications prosodiques.

Le document de Véronique Aubergé \cite{auberge2002gestalt} nous offre une approche par la méthode Gestalt de la prosodie.

PSOLA (Pitch Synchronous Overlap and Add) est une technique de traitement de signal permettant d'effectuer des traitements sur les discours et, couplée, à un module pour les transformations spectrales peuvent permettre un système de conversion de voix \cite{valbret1992voice}.

\section{Références sur l'expression faciale}\label{ref_transfo_faciales}

Par rapport à un certain discours, les expressions faciales et particulièrement buccales peuvent être régies par plusieurs règles de synchronisation afin de pouvoir par exemple améliorer un discours homme-machine \cite{beskow1995rule} .

La prosodie faciale permet, en outre, d'aider à la détection ethnique du protagoniste, qui est un facteur intéressant pour notre projet car il porte effectivement sur trois langues : le français, l'anglais (américain) et le japonais\cite{matsumoto1992american}.


\section{État de l'art}\label{state_of_the_art}

La prosodie permet, par exemple d'aider, à la reconnaissance des différents dialectes arabes. En d'autres termes, malgré une proximité géographique et linguistique, la prosodie permet la différenciation \cite{rouas2006identification}.

L'apprentissage de la prosodie, n'est pas un desir extremement nouveau, ainsi nous nous somme documenté sur l'état actuel de l'art en matière de logiciel pour l'apprentissage de la prosodie, et avont trouvé certains documents proche de ce que l'on cherche à développer \cite{10.4000/alsic.332}. 

\section{Programmation avec du contenu vidéo}

Une vidéo, ne se réduit pas seulement à un format, elle se compose également des \textit{Codecs}, ces dispositifs permettent la compression/decompression d'un signal numérique. Ces \textit{Codecs} sont des mecanismisme importants lorsque l'on veut faire du traitement sur une vidéo car c'est cette technique qui permet l'encodage et le décodage de fichier afin de pouvoir les lire \cite{ghanbari1999video} \cite{he2013introduction}.
\section{Ergonomie}
Pour désigner notre application, tout ne peut se faire à notre guise, des règles et des principes existent afin de mêler harmonieusement ergonomie et design \cite{lente2014scenariser}.

\section{Base de données}

Durant ce projet, il a fallu trouver le système de base de données le plus adapté à nos besoins. Pour ce faire, nous nous sommes appuyés sur l'article \cite{strauchnosql}, disséquant l'architecture \textit{NoSQL} (Not Only SQL).
Nous tournant finalement vers le \textit{SGBDR} (Système de Gestion de Base de Données Relationnelles) \textit{SQLite}, nous nous sommes informés sur l'utilisation et les finalités de ce logiciel \cite{kreibich2010using}.

