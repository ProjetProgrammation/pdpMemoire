\chapter{Analyse de l'existant}

\section{Compréhension du sujet}\label{comprehension_sujet}

La prosodie est un terme mettant en exergue les liens entre les expressions (faciales et vocales) et les émotions qui en découlent. Néanmoins, nous ne pouvons réduire cette définition à ce simple résumé, de ce fait l'article \cite{bachorowski1999vocal} nous permet d'obtenir une définition plus précise sur ce terme technique.

Ainsi, la voix est un élément important de la prosodie car elle est un acteur primordial de la transmission d'émotions mais en fonction de nos origines socio-culturelles cette perception peut différer \cite{auberge2002prosodie}.
De plus, la voix se combine avec les expressions faciales qui sont le portrait de l'émotion que nous voulons transmettre. Les changements que nous effectuons sur notre visage(position des sourcils, lèvres...) permettent l'expression d'émotions par celui-ci \cite{ekman2003unmasking} .

Enfin, le document \cite{fourer:hal-00992083} nous offre une approche sociétale du problème avec une étude sur la prosodie attitudinale concernant la langue japonaise.


\section{Références sur la transformation de la voix}\label{ref_transfo_voix}

Pour les études prosodiques, des modulations, des découpages et des modifications de la voix sont nécessaires. Cependant, cette dernière est un mécanisme complexe comme le présent le cours de Ricardo Gutierrez-Osuna \cite{Gutierrez-Osuna:ISP-PMS} et l'article de l'IEEE Signal Processing Letters \cite{haagen1994transformation}.

L'extrait de la IEEE International Conference on Acoustics de 1998 \cite{Acero:ICASSP98-II-881} décrit deux possibles modifications prosodiques.

La méthode Gestalt est une méthode de psychothérapie qui s'appuie sur la manière dont les personnes entrent en contact, se comportent dans une relation. C'est une methode qui peut s'appliquer à notre projet \cite{auberge2002gestalt}.

PSOLA (Pitch Synchronous Overlap and Add) est une technique de traitement de signal. Associée à un module, elle permettrait d'effectuer des transformations spectrales pour modifier la voix \cite{valbret1992voice}.

\section{Références sur l'expression faciale}\label{ref_transfo_faciales}

Par rapport à un discours précis, les expressions faciales, et particulièrement buccales, peuvent être régies par plusieurs règles de synchronisation afin de pouvoir, par exemple, améliorer un discours homme-machine \cite{beskow1995rule} .

La prosodie faciale permet, en outre, d'aider à la détection ethnique du protagoniste, qui est un facteur intéressant pour notre projet car il porte sur quatre langues : le français, l'anglais (américain), le portugais du Brésil et le japonais\cite{matsumoto1992american}.


\section{État de l'art}\label{state_of_the_art}

La prosodie permet, par exemple, d'aider à la reconnaissance des différents dialectes arabes. En d'autres termes, malgré une proximité géographique et linguistique, la prosodie permet la différenciation\cite{rouas2006identification}.

L'apprentissage de la prosodie n'est pas un désir extrêmement nouveau. Ainsi nous nous sommes documentés sur l'état actuel de l'art en matière de logiciel concernant l'apprentissage de la prosodie et nous avons trouvé certains documents proches de ce que l'on désirait développer \cite{10.4000/alsic.332}. 

\section{Programmation avec du contenu vidéo}

Une vidéo ne se réduit pas seulement à un format, elle se compose aussi des \textit{Codecs}. Ces dispositifs permettent la compression/decompression d'un signal numérique. Ces \textit{Codecs} sont des mécanismes importants lorsque l'on veut faire du traitement sur une vidéo. C'est effectivement cette technique qui permet l'encodage et le décodage de fichier afin de pouvoir les lire \cite{ghanbari1999video} \cite{he2013introduction}.
\section{Ergonomie}
Pour designer notre application, nous devons suivre des règles et des principes existants afin de mêler harmonieusement ergonomie et design \cite{lente2014scenariser}.

\section{Base de données}

Durant ce projet, il a fallu trouver le système de base de données le plus adapté à nos besoins. Pour ce faire, nous nous sommes appuyés sur l'article \cite{strauchnosql}, disséquant l'architecture \textit{NoSQL} (Not Only SQL).
Nous nous sommes finalement tournés vers le \textit{SGBDR} (Système de Gestion de Base de Données Relationnelles) \textit{SQLite} et nous sommes informés sur l'utilisation et les finalités de ce logiciel \cite{kreibich2010using}.

