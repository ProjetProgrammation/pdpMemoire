\chapter{Analyse de l'existant}



\section{Compréhension du sujet}\label{comprehension_sujet}

Afin de comprendre au mieux la prosodie, l'article \cite{bachorowski1999vocal} nous permet de mettre une définition sur ce terme technique.

L'article \cite{auberge2002prosodie} nous donne des notions sur les émotions que l'on peut passer avec la voix, alors que le livre \cite{ekman2003unmasking} nous aide sur les expressions faciales.

Enfin, le document \cite{fourer:hal-00992083} nous offre une approche sociétale du problème avec une étude sur la prosodie attitudinale pour la langue japonaise.


\section{Références sur la transformation de la voix}\label{ref_transfo_voix}

Le cours de Ricardo Gutierrez-Osuna \cite{Gutierrez-Osuna:ISP-PMS} et l'article de l'IEEE Signal Processing Letters \cite{haagen1994transformation} sont des descriptions physiques de la modulation et modification de la voix.

L'extrait de la IEEE International Conference on Acoustique de 1998 \cite{Acero:ICASSP98-II-881} décrit deux possibles modifications prosodiques.

Le document de Véronique Aubergé \cite{auberge2002gestalt} nous offre une approche par la méthode Gestalt de la prosodie.

Dans cet article du journal Voice transformation using PSOLA technique \cite{valbret1992voice}, un système de conversion de voix utilisant PSOLA et un module pour les transformations spectrales sont étudiés.

\section{Références sur l'expression faciale}\label{ref_transfo_faciales}

Le document \cite{beskow1995rule} expose une approche sur la définition de règles de synchronisation des expressions faciales et particulièrement buccales par rapport à un certain discours.

Le document \cite{matsumoto1992american} est utile dans le sens qu'il porte sur l'étude de la prosodie faciale, et ce afin de pouvoir détecter l'ethnie du protagoniste, ce qui nous est utile car le projet porte sur trois langues : le français, l'anglais (américain) et le japonais.


\section{État de l'art}\label{state_of_the_art}

L'article \cite{rouas2006identification} nous donne une ligne directrice pour tout ce qui touche la reconnaissance des différents dialectes arabes. En d'autres termes, malgré une proximité géographique et linguistique, la prosodie permet la différenciation.
L'article de la revue en ligne Alsic \cite{10.4000/alsic.332} nous apporte un point sur l'état actuel de l'art en matière de logiciel pour l'apprentissage de la prosodie, ce qui est proche de ce que l'on cherche à développer.

\section{Programmation avec du contenu vidéo}

Les articles \cite{ghanbari1999video} et \cite{he2013introduction} nous ont permis de mieux comprendre le concept d'encodage des vidéos et l'utilisation de \textit{Codecs} pour pouvoir décoder ces vidéos afin de les lire.

\section{Ergonomie}

L'article \cite{lente2014scenariser} est une étude sur la “collaboration entre Ergonomie, Design et Ingénierie”, qui nous offre une façon de procéder afin de réaliser une ergonomie des plus efficace.

\section{Base de données}

Durant ce projet, il a fallu trouver le système de base de données le plus adapté à nos besoins. Pour ce faire, nous nous sommes appuyés sur l'article \cite{strauchnosql}, disséquant l'architecture \textit{NoSQL} (Not Only SQL).
Nous tournant finalement vers le \textit{SGBDR} (Système de Gestion de Base de Données Relationnelles) \textit{SQLite}, nous nous sommes informés sur l'utilisation et les finalités de ce logiciel avec le livre \cite{kreibich2010using}.

